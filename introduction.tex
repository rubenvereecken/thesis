\pagenumbering{arabic}
\chapter{Introduction}

\label{introduction}
Reinforcement Learning is what we have labeled the remarkable ability of man and beast alike
to learn from their surroundings and their interactions with it.
It is embodied in our learning from past mistakes or our failing to do so.
For quite some time now people have attempted to bestow this same ability on
our computing devices so that they would figure out how to tackle problems by themselves
without explicit instructions.

\paragraph{}
Autonomous learning agents are already present
in the periphery of our day to day activities.
For example,
networks can learn to route smartly
depending on time of day
and congestion
or any other available information.
More and more,
autonomous learners are even invading
the commercial and personal space.
Smart homes can learn to anticipate the inhabitants'
behavior and proactively control the thermostat,
leading to a better experience
and lessening the ecological footprint of a home
without the intervention of humans.

Recent years have also seen a rise of interest
in robotics and control,
a popular example of which is the ongoing research
concerning self-driving cars
\parencite{selfdriving}
along with the multitude of uses for an autonomous robot
even if only in a restricted and controlled environment.

\paragraph{}
Clearly,
in this day and age
where progress is considered
one of the most noble of goals,
reinforcement learning could conceivably
offer the tools
for a great many things
we might consider \textit{of the future}.

The field is relatively young
yet has already booked remarkable advances.
Still there is a very real sense of unfinishedness.
I do not hint at immaturity
for it has already reached the point
at which it can be considered mature
but rather that the field has not reached its full potential.
The applications that have been conceived successfully
are not as widespread as they could be
and those applications we aim to achieve
still partially elude us.
Of course, for a researcher
this poses incredibly interesting opportunities.

\paragraph{}
As stated,
the field of reinforcement learning
inches forward and manages the occasional leap.
In recent years,
some of the greatest contributions
make use of Deep Learning
within the reinforcement learning context.

Deep Learning originates in supervised learning
and concerns itself with learning
end-to-end, hierarchical representations of data
which can manage unprecedented performance
and are especially interesting
because they can eliminate the need for handcrafted features.
A great overview of
some of the most important contributions
and noteworthy applications
in this regard is given by
\citeauthor{Y.2015a} (\citeyear{Y.2015a}).

The field has had successes in
face recognition
\parencite{lawrence1997face,wolf2014deepface},
speech recognition
\parencite{dahl2012context,graves2013speech}
and
genomics
\parencite{leung2014deep,xiong2015human}
and is far from limited to these alone.

\paragraph{}
Recent successes in reinforcement learning
include successfully transferring deep learning principles
outside of their original context.

A great effort has recently been put forward by
\citeauthor{Mnih2015} (\citeyear{Mnih2013,Mnih2015})
with the introduction of the Deep Q-Network.
Through the combination of great research and sufficient computation power
their learning agents have managed to beat humans
on an array of Atari games.
Still, some games remain largely or completely unbeaten.
The scientific community jumped
on the opportunity to further improve on DQN
and to explore the area
that now goes by the name of deep reinforcement learning.

\paragraph{}
The current state of the art reinforcement learning
can now learn straight from raw image pixels
given a sufficiently deep neural network
to represent its value function.
While spatial features can be learned hierarchically,
the dimension of time has been mostly neglected or simplified.
% informal rep of mdp
This is because all theoretical work is based
on the requirement that all state information that
could be useful is present;
no hidden state allowed,
not even historical state that is not observable anymore.

Experiments with Atari games
manage to circumvent this requirement somewhat
since all games are too simple
to contain such long-term dependencies.
All information one could need is usually encompassed
within a few frames at any point,
like the velocity of a moving object on the screen
can be calculated from two frames.

\paragraph{}
This neglect and oversimplification of the dimension of time
struck me as interesting and as an area
deserving of investigation.
In this thesis,
I set out to compare different approaches
to handling the dimension of time in reinforcement learning.

\section{Research Question}
\label{sec:research_question}
In this thesis I set out to investigate
what the effect of time dependencies is
in a reinforcement learning setting
and how to successfully learn these
in order to learn control policies.
In particular,
the problems I consider
are all Atari games
where the agent needs to decide on an action
given solely raw pixels as input
along with a game-specific reward signal
to guide its actions.
What makes these games interesting is that sometimes
state is not available within a single frame;
one does not know which way the laser is moving
from just looking at a standstill.

\paragraph{}
The current standard approach to cope with
features that can be derived over time such as velocity
is to feed multiple frames into the architecture at once.
This already is an improvement over using just a single frame
because the network suddenly has more information available.
The idea behind passing in multiple frames
is that the network will learn to combine the information
available in multiple time steps.

\paragraph{}
I will consider three alternative approaches.
Late Fusion and 3D convolutional networks
are of the same variety:
they too employ a fixed window but
do so differently.
Both have successfully been applied in a machine learning context before
\parencite{Karpathy2014,Ji2013}.
The last and third approach involves
long short-term memory,
which has already been shown to be able to remember
dependencies over arbitrary amounts of time
\parencite{Bakker2001},
albeit it in a vastly more simple setting.

\paragraph{}
My contribution consists
of applying these approaches
in a deep reinforcement learning setting
and comparing them in different scenarios
against one another and against the standard approach.

\section{Outline}
\label{sec:outline}
This thesis will gradually build up sufficient understanding
for the reader to understand the contributions presented.
Chapter 2 concerns itself with providing background
concerning Machine Learning.
Following with Artificial Neural networks is Chapter 3,
which examines how to construct and train neural networks.
Chapter 4 will provide a full understanding of
the reinforcement learning setting,
what is aimed to being achieved
and how neural networks fit in.
Chapter 5 follows naturally
with more recent extensions
to reinforcement learning
and gives some necessary background
to deep reinforcement learning.

\paragraph{}
The contributions of this thesis are mainly in Chapters 6 and 7.
Chapter 6 outlines 3 architectures
that are each in turn explored and discussed.
At the end of Chapter 6 the reader can find an overview
comparing these architectures with the standard approach
and one another.

Chapter 7 will draw conclusions,
outline my contributions
and point to future work.

\paragraph{}
Appendix A can be visited for experimental details
and statistical tests
while Appendix B
contains additional experiments,
not all of which are mentioned in the text.
