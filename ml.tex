\chapter{Machine Learning}

\section{Introduction}
The world of today is generating huge amounts of data
as it goes along,
and more and more of it is put to use.
It's quite probable that any consumer in our society
that goes shopping,
uses some kind of service,
or uses the internet
on a daily basis
performs actions that end up in a database.
Not only do persons generate data,
so does everything else we monitor,
and our society happens to monitor quite a lot.
This data is invaluable for any organization that benefits
from information,
and truly, most do.
It is used by financial institutions like insurance companies,
medical institutions, commercial companies, scientific research
and everything in between.
Evidently, as the need for knowledge (or awareness thereof)
increases, so must our abilities to uncover it,
making machine learning an incredibly interesting topic
in today's world.


People are good at creating hypotheses
about what things mean,
connecting the dots
and predicting cause or outcome.
Yet we can only cope
with so much data at a time
and without direction,
the answers we seek may elude us.
Enter the computing device,
programmable to our needs.
Computer programs often take the form
of our expert knowledge put to detailed writing
but they can also cover the gaps where
we don't have the knowledge
and instead need to uncover it.
This includes anything of valuable information to us,
be it patterns in certain data or hypotheses for predictions.
Where the domain expert describes such hypotheses
using a computer program,
the machine learning expert devises ways of uncovering
the hypotheses.
Evidently, the power of computers would be much greater
if they would be made able to learn as we do.

Great advances have already been made in the domain,
yet we have not managed to make computers learn
as well as humans do,
at least not in as many regards as we do.
The best applications do, however,
get better results than humans in very specific domains.


\paragraph{}
Machine learning is a collection of methods
used to bridge the aforementioned gap
between data and knowledge.
A machine learning algorithm learns from data
and learns an hypothesis or model on it,
which can then be used to make predictions on future data.
This separates it from other algorithms
designed by an expert
that follow a static rule set.
It is inherently an interdisciplinary field,
with its roots firmly embedded
in a statistical foundation
combined with AI,
drawing inspiration from fields such as
information theory,
complexity theory,
psychology
and other fields.

The rest of this chapter will describe the basics of machine learning
along with the specific tools I will use throughout this thesis.
The foundation should be quite sufficient to build up
an understanding of the field
so as to allow the reader to follow the following chapters comfortably.
Next to exploring a foundation,
I will describe specific methods such as neural networks
and convolutional layers,
both exciting techniques that will be used further on.
However, perhaps to the reader's delight,
a cursory glance at the high-level introduction of these methods
should suffice for the reader who only seeks to understand
the applications described herein.
The careful reader is of course cordially invited
to read the more detailed descriptions.

