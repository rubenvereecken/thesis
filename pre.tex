\renewcommand{\abstractname}{
  \centering\Large\rm
  Abstract
}
\begin{abstract}
  Deep learning is an increasingly popular topic
  in the area of machine learning.
  Reinforcement learning has recently adopted it
  as a method for learning behavior
  from raw sensory input
  such as image frames.
  Good results have already been achieved
  but many problems still elude us,
  % Maybe mention POMDP
  especially where dependencies over time
  must be learned in order to achieve results.

  \paragraph{}
  In this thesis,
  we adopt the Arcade Learning Environment
  to train agents
  that learn how to play Atari games
  based solely on raw pixel input
  and a reward signal.
  In order to be successful
  these agents need to learn
  important features
  autonomously,
  not only visually
  but also temporally.

  \paragraph{}
  The current state of the art,
  Deep Q-Network (DQN),
  only takes into account the most recent frames.
  We explore three alternative neural network architectures
  and their capabilities of handling
  dependencies over time
  and compare these
  against DQN.
  % The first is the Late Fusion network
  % which aims to encompass time information
  % by learning from two input frames
  % that correspond to different time steps.

  \forceindent
  Two of the networks considered here
  are valid alternatives for learning
  short-term, fixed-time
  features
  just like humans can intuitively
  derive the direction something is moving in.
  The first aims to encompass time information
  by learning from two input frames
  that correspond to different time steps
  and merging that information in a deeper layer.

  \forceindent
  The second involves convolutions over time
  and is shown to significantly outperform the standard approach
  on a problem
  with considerable amounts of obscured information.

  % The second is a 3D convolutional network
  % which allows the network designer
  % a more fine-grained control
  % over how time information is merged
  % throughout the network.

  \forceindent
  The last architecture under consideration
  makes use of
  long short-term memory,
  which effectively equips the network
  with internal memory
  that can hold past information.
\end{abstract}

\renewcommand{\abstractname}{
  \centering\Large\rm
  Acknowledgements
}
\begin{abstract}
  First and foremost I would like to thank
  Prof. Dr. Peter Vrancx
  for the countless times we have sat together,
  brainstorming for interesting problems
  and discussing how to approach them.
  His expertise always managed to guide me in the right direction
  and his patience never failed to baffle.
  This thesis would not have existed as it does
  without the inspiration offered
  as well as all the invaluable advice
  he was always willing to impart.

  Secondly,
  I would also like to thank
  Prof. Dr. Ann Now\'{e}
  for the opportunity.

  Lastly,
  I would like to thank my mother
  for her endless support
  and bottomless faith
  in anything I have ever undertaken,
  not the least of which
  this dissertation.

\end{abstract}
